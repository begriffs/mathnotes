\documentclass[letterpaper]{article}
\usepackage{amsfonts}
\usepackage{amsmath}
\usepackage{amssymb}
\newtheorem{theorem}{Theorem}[section]
\newtheorem{lemma}[theorem]{Lemma}
\newtheorem{proposition}[theorem]{Proposition}
\newtheorem{corollary}[theorem]{Corollary}

\newenvironment{proof}[1][Proof]{\begin{trivlist}
\item[\hskip \labelsep {\bfseries #1}]}{\end{trivlist}}
\newenvironment{definition}[1][Definition]{\begin{trivlist}
\item[\hskip \labelsep {\bfseries #1}]}{\end{trivlist}}
\newenvironment{example}[1][Example]{\begin{trivlist}
\item[\hskip \labelsep {\bfseries #1}]}{\end{trivlist}}
\newenvironment{remark}[1][Remark]{\begin{trivlist}
\item[\hskip \labelsep {\bfseries #1}]}{\end{trivlist}}

\begin{document}
\title{Abstract Algebra\\
Math 742}
\author{Joe Nelson \and Scott Pellicane}
\date{Spring 2010}
\maketitle

\section{January 20}

All the rings in this course have a multiplicative identity.

\begin{theorem}
Any integral domain has a field of fractions.
\end{theorem}

\section{January 22}

\section{January 25}

\begin{lemma}
Any polynomial ring with coefficients in a field is Euclidean with the degree function.
\end{lemma}

\begin{theorem}
A Euclidean domain is a PID.
\end{theorem}

\begin{lemma}
In an integral domain(?) primes are irreducible.
\end{lemma}

\section{January 27}

\begin{lemma}
If a GCD exists, it is unique up to associates. Also associates of GCDs are GCDs.
\end{lemma}

\begin{lemma}
Any two nonzero elements $a$ and $b$ of a PID have a GCD, $(a, b)$. Moreover, there are elements $x$, $y$ in the PID such that $(a, b) = ax + by$. 
\end{lemma}

\begin{lemma}
In a PID, primes and irreducibles are the same.
\end{lemma}

\begin{lemma}
Prime factorization in an integral domain is unique up to associates.
\end{lemma}

\begin{theorem}
PIDs are UFDs.
\end{theorem}

\section{January 29}

\begin{lemma}
\label{prim}
In a polynomial ring with coefficients in a UFD, any product of primitives is primitive.
\end{lemma}

\begin{lemma}
Let $R$ be a UFD, and $F$ its field of fractions.
\begin{enumerate}
\item Suppose $f \in R[x]$ factors as a product of two elements in $F[x]$. Then it factors as two  elements in $R[x]$. Further, the degrees of the factors in $R[x]$ match those in $F[x]$.
\item If $p \in R$ is prime, then so is the constant polynomial $p$ in $R[x]$.
\item If a primitive polynomial in $R[x]$ is prime in $F[x]$ then it is prime in $R[x]$.
\end{enumerate}
\end{lemma}


\section{February 8}

\begin{theorem}
If $R$ is UFD then so is $R[x]$.
\end{theorem}

\begin{corollary}
If $R$ is UFD then so is $R[x_1, x_2, \ldots, x_n]$.
\end{corollary}

\begin{theorem}
Let $R$ be a UFD and $f = (a_i) \in R[x]$. If $p$ is a prime in $R$ such that $p$ divides $a_i$ for $0 \leq i \leq n-1$, and $p$ does not divide $a_n$ and $p^2$ does not divide $a_0$, then $f$ is prime in $F[x]$ where $F$ is the field of fractions of $R$.
\end{theorem}

\section{February 10}

\begin{lemma}
Let $F \supset E \supset K$ be fields. Then $|F \colon K| = |F \colon E||E \colon K|$.
\end{lemma}

\begin{theorem}
Nonzero nonconstant polynomials have roots. Specifically, let $K$ be a field and $f \in K[x]$. Then there exists an $\alpha$ in a finite extension of $K$ such that $(\alpha)f = 0$.
\end{theorem}

\section{February 12}

\begin{lemma}
Let $f \in K[x]$ where $K$ is a field.  Then there exists an extension of $K$ that contains a root of $f$.
\end{lemma}

\begin{theorem}
Let $f \neq 1$ be monic in $K[x]$ where K is a field.  Then $f$ has a splitting field $F$ such that the degree of the extension $|F \colon K|$  divides $n!$.
\end{theorem}

\begin{theorem}
Let $\theta$ be a field isomorphism between $F$ and $\bar{F}$. Then $\theta$ extends to an isomorphism between the splitting fields of $f \in F[x]$ and $(f) \theta \in \bar{F}[x]$ for any $f \in F[x]$.
\end{theorem}

\section{February 15}

\begin{lemma}
A polynomial $f \in K[x]$ has a root with multiplicity in some field extension of $K$ iff $gcd(f, Df) \ne 1$.
\end{lemma}

\begin{corollary}
An irreducible polynomial $f$ has a double root in some field extension iff $Df = 0$.
\end{corollary}

\begin{corollary}
Let $f \in K[x]$ be irreducible. Then $f$ has a double root in an extension field unless
\begin{enumerate}
\item $char K > 0$ and
\item $f \in K[x^p]$ and
\item $K$ is not perfect
\end{enumerate}
\end{corollary}

\section{February 17}

\begin{lemma}
Every finite field has prime-power order.
\end{lemma}

\begin{lemma}
All finite fields of a given size are isomorphic.
\end{lemma}

\begin{lemma}
There exists a finite field of size $p^n$ for any $n \ge 0$ and prime $p$.
\end{lemma}

The unique finite field of size $p^n$ is denoted $GF(p^n)$.

\begin{lemma}
Suppose a finite abelian group has at most one subgroup of prime order for any prime. Then the group is cyclic.
\end{lemma}

\section{February 19}

\begin{theorem}
\emph{\textbf{(Primitive Element)}}
Let $F \supset K$ be a finite field extension. Then $F = K[\alpha]$ for some $\alpha$ iff there are only finitely many intermediate fields between $F$ and $K$.
\end{theorem}

\begin{lemma}
The $n$th roots of unity in a field form a cyclic multiplicative group of order $n$.
\end{lemma}

Define $\Phi_n = \displaystyle\prod_{o(a) = n} (x - a) \in F[x]$.

\section{February 22}

\begin{lemma}
The polynomial $x^n - 1$ has $n$ distinct roots in a field of characteristic 0 or $p$ where $p$ does not divide $n$.
\end{lemma}

\begin{lemma}
$x^n - 1 = \displaystyle\prod_{d | n} \Phi_d$.
\end{lemma}

\begin{corollary}
$\Phi_n \in \mathbb{Z}[x]$.
\end{corollary}

\begin{theorem}
\emph{\textbf{(Gauss)}}
$\Phi_n$ is irreducible over $\mathbb{Q}[x]$ for all $n$.
\end{theorem}

\section{February 24}

\begin{lemma}
\emph{\textbf{(Dedekind)}}
Let $M$ be a multiplicative group, $F$ a field, and let $\lambda_i \colon M \rightarrow F^{\times}$ for $1 \leq i \leq n$ be distinct homomorphisms. If $a_i \in F$ for $1 \leq i \leq n$ and $\sum a_i \lambda_i (x) = 0$ for all $x \in M$, then $a_i = 0$ for $1 \leq i \leq n$.
\end{lemma}

\begin{corollary}
Let $\sigma_i$ for $1 \leq i \leq n$ be distinct automorphisms of a field $F$. If $a_i \in F$ for $1 \leq i \leq n$ and $\sum a_i \sigma_i (x) = 0$ for all $x \in F$, then $a_i = 0$ for $1 \leq i \leq n$.
\end{corollary}

\begin{definition}
Let $S$ be a set of automorphisms of a field $F$.  The field $F^S = \left\{x \in F \colon x^{\sigma} = x \textnormal{for all} \sigma \in S\right\}$ is called the \emph{fixed field of S}.
\end{definition}

\begin{lemma}
$|F \colon F^S| \geq |S|$
\end{lemma}

\begin{lemma}
Let $G$ be a finite group of automorphisms of a field $F$.  The trace map $tr \colon F \rightarrow F^G$ defined by $x \mapsto \sum_{\sigma \in G} x^\sigma$ is a surjective linear transformation of $F^G$.
\end{lemma}

\begin{lemma}
Let $|G| < \infty$. Then $|F \colon F^G| \leq |G|$.
\end{lemma}


\end{document}
