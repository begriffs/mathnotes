\documentclass[letterpaper]{article}
\usepackage{amsfonts}
\usepackage [intlimits]{amsmath}
\usepackage{amssymb}
\usepackage{colonequals}
\newtheorem{theorem}{Theorem}[section]
\newtheorem{lemma}[theorem]{Lemma}
\newtheorem{proposition}[theorem]{Proposition}
\newtheorem{corollary}[theorem]{Corollary}

\newcommand{\mdot}{{\cdot}}

\newenvironment{proof}[1][Proof]{\begin{trivlist}
\item[\hskip \labelsep {\bfseries #1}]}{\end{trivlist}}
\newenvironment{definition}[1][Definition]{\begin{trivlist}
\item[\hskip \labelsep {\bfseries #1}]}{\end{trivlist}}
\newenvironment{example}[1][Example]{\begin{trivlist}
\item[\hskip \labelsep {\bfseries #1}]}{\end{trivlist}}
\newenvironment{remark}[1][Remark]{\begin{trivlist}
\item[\hskip \labelsep {\bfseries #1}]}{\end{trivlist}}

\begin{document}
\title{Topology\\
Math 751}
\author{Joe Nelson}
\date{Fall 2010}
\maketitle

These notes contain what I consider the highlights and subtle points
of Professor Laurentiu Maxim's lectures. I express the concepts in
my own words, and insert, expand, and rearrange topics.

This document is distributed under the Creative Commons 3.0 attribution
license.

\section{Lecture 1 -- September 3}

\subsection{Paths and a topological invariant}

Invariants of topological spaces are properties preserved by
homeomorphisms. Invariants often help distinguish one topological
space from another, but they have other uses as lemma 1.1 shows.

Let $I = [0, 1] \subset \mathbb{R}$. Recall that a \emph{path} from
points $x$ to $y$ in a topological space $X$ is a continuous map
$f \colon I \rightarrow X$ such that $f(0) = x$ and $f(1) = y$. The
set of such maps is denoted $P(X, x, y)$.

Define the relation $x \sim y$ if $P(X, x, y) \ne \emptyset$. It
is in fact an equivalence. First, $x \sim x$ for any $x$ by the
constant path. If $x \sim y$ by a path $f$, then $y \sim x$ by $t
\mapsto f(1-t)$. We can call this ``reversed'' path $f^{-1}$, taking
care not to confuse it with an inverse map. Finally suppose $x \sim
y$ by $f$ and $y \sim z$ by $g$. For any $k \in (0, 1)$ define the
\emph{k-composition} of paths by
\[ f \ast_k g \colon t \mapsto
   \left\{
     \begin{array}{lr}
       f(\frac{t}{k}) & t \in [0, k]\\
       g(\frac{t-k}{1-k}) & t \in [k, 1]
     \end{array}
   \right.
\]
Now $f$ and $g$ agree at $t = k$ and each is continuous, so $f
\ast_k g$ is continuous by the pasting lemma. This puts the composition
in $P(X, x, z)$. (When the subscript $k$ is omitted, it is assumed
$k = \frac{1}{2}$.) Hence $x \sim z$.

A \emph{path component} of $X$ is an equivalence class of this
relation. For any topological space $X$ let $n(X)$ be the number
of path components of $X$.

\begin{lemma}
If $f \colon X \rightarrow Y$ is surjective and continuous then
$n(X) \geq n(Y)$.
\end{lemma}
\begin{proof}
Suppose a path $p$ connects points $a$, $b \in X$. A composition
of continuous functions is continuous, so $p \circ f$ is a path
which connects $f(a)$ and $f(b)$. Now choose a representative point
from each path component of $Y$. By the surjectivity of $f$, there
exist points in the preimage of each representative. These points
are each in distinct path components of $X$ by the contrapositive
of the preceeding, so $X$ has as many path components as $Y$.
\end{proof}

\begin{corollary}
The $n$ function is a topological invariant
\end{corollary}

This invariant, or more precisely the inequality in the lemma,
provides an elementary way to understand and prove a theorem from
calculus. Here is the usual proof without an invariant.

\begin{theorem}
If $f \colon I \rightarrow I$ is continuous then it fixes a point.
\end{theorem}
\begin{proof}
WLOG $f(0) > 0$ and $f(1) < 1$ or we would be done. Let $g(x) =
f(x) - x$. Now $g$ is continuous and $g(0) > 0$ while $g(1) < 0$.
By the intermediate value theorem, $g$ attains zero at some point,
which is a fixed point for $f$.
\end{proof}

Intuitively this proof works because the graph of $f$ in its course
across the unit square must somewhere cross the diagonal line which
runs through $(0, 0)$ and $(1, 1)$, The invariant $n$ affords us a
proof which shares the same motivation while avoiding the intermediate
value machinery.

\begin{theorem}
If $f \colon I \rightarrow I$ is continuous then it fixes a point.
\end{theorem}
\begin{proof}
Assume not. Then $f(x) \neq x$ for any $x$ and it is sensible to
define $r \colon I \rightarrow \{\pm 1\}$ by \[ r(x) = \frac{f(x)
- x}{|f(x) - x|} \] Also, $f(0) > 0$ so $r(0) = 1$ and $f(1) < 1$
so $r(1) = -1$. Hence $r$ is surjective. But notice the $n$-values
of the spaces. First, $n(I) = 1$ because for any $a < b$, the
identity function on $[a, b]$ is a path connecting them. The discrete
space $\{\pm 1\}$ is totally disconnected so $n(\{\pm 1\}) = 2$.
The lemma above implies $2 \leq 1$, a contradiction.
\end{proof}

\section{Lecture 2 -- September 8}

\subsection{The fundamental group}

Another (important) invariant is a group made from the paths of a
space and the $\ast$ operation. To motivate its definition we will
begin naively. Is the set of all paths on a space a group under
$\ast$? If so, then $p \ast p$ must be defined for every path $p$.
However, composition of paths is only defined when one path ends
where the other begins, so $p$ must be a \emph{loop} (a path starting
and ending at the same point). Similarly, to be sensibly composed,
arbitrary loops must originate from the same point.

Take two: for an arbitrary $x \in X$, is $\Omega(P, x) = P(X, x,
x)$ a group under $\ast$? The first problem is that associativity
fails for the majority of loops. In $(f \ast g) \ast h$, the loop
$f$ is compressed to a quarter of its original ``length'' and $h$
is compressed only a half, whereas in $f \ast (g \ast h)$ exactly
the opposite is true. If $\ast$ could selectively behave like
$\ast_k$ for various $k$ then associativity would work, since $f
\ast_\frac{1}{3} (g \ast h) = (f \ast g) \ast_\frac{2}{3} h$.

Another problem is finding an identity element. Even an innocent
guess like the constant loop can't undo the compression that paths
undergo through composition.

Although this naive approach fails, it seems well-intentioned. If
we can find a precise way to be less fussy about composition
compression, we can make it work.

The solution is a concept called \emph{homotopy}. A homotopy is a
path in the function space $P(X, x, y)$ i.e. a ``path path.'' Two
paths connected by a homotopy are said to be \emph{homotopic} and
path components of $P(X, x, y)$ are called \emph{homotopy classes}.
The following lemma illustrates a nice property of these classes.

\begin{lemma}
If $f$, $g \in \Omega(X, x)$ and $k$, $k' \in (0, 1)$ then $[f
\ast_k g] = [f \ast_{k'} g]$.
\end{lemma}
\begin{proof}
Define $\phi(s) = (1 - s)k + sk'$ and $F(s) = f \ast_{\phi(s)} g$.
Now $F(0) = f \ast_k g$ and $F(1) = f \ast_{k'} g$. Show $F$ is
continuous (need definition of open sets in function space). Thus
$F$ is a homotopy and the proof is done.
\end{proof}

\end{document}
